% LaTeX resume using res.cls
\documentclass[line,margin]{res} 
%\usepackage{helvetica} % uses helvetica postscript font (doucture helvetica.sty)
%\usepackage{newcent}   % uses new century schoolbook postscript font 
%\usepackage{setspace}
\usepackage{url}
\addtolength{\textheight}{70pt}

\begin{document}

\name{Jia Sun}
% \address used twice to have two lines of address
\address{7309 Montclair St, Burnaby, BC, Canada, V5A 3J2}
\address{+1-778-881-3070 \\jias@sfu.ca}

\begin{resume}

\section{Education} 
    {\sl Dual Degree Program Candidate for Bachelor of Computing Science}  
      \begin{itemize} \itemsep -2pt
        \item[-] Simon Fraser University, Canada \hfill Sep 2009 - Aug 2012 (expected)
        \item[-] Zhejiang University, China \hfill Sep 2007 - Aug 2009
      \end{itemize}
	  \vspace{-1.1em}
      Two Bachelor Degrees from both Simon Fraser University and Zhejiang University

\section{Research Experience}
    {\sl Research Assistant} \hfill Jan 2011 - Apr 2011 \\
	{\sl Visualization for Nursing Home Fall Detection}\\
	Vision and Media Lab, Simon Fraser University, Canada
	\begin{itemize} \itemsep -2pt
	    \item[-] Developed a system to visualize nursing home falls from surveillance videos
		\item[-] Three methods developed, one concentrated on color, one used fast fowarding, and the other assembled different fall clips into an integrated scene
	\end{itemize}

	{\sl Research-Based Graduate Course} \hfill Jan 2011 - Apr 2011 \\
	{\sl Performance Comparison of MPI Workloads on HPC clusters versus Amazon EC2} \\
	Mutlicore System, Simon Fraser University, Canada
	\begin{itemize} \itemsep -2pt
	    \item[-] Designed benchmark sets to run on both academic HPC clusters and Amazon EC2 Cluster Compute Instances to compare the performance
		\item[-] Concluded with node compute capacity has consistent impact on performance, while interconnection has larger impact on cross communication tasks
		\item[-] EC2 Cluster Compute Instances can be competitive with the HPC cluster, at least for median sie CPU-intensive MPI applications
	\end{itemize}

    
\section{Academic Projects}
    \begin{itemize} \itemsep -2pt 
	  \setlength{\itemindent}{-1.5em}
	    \item[-] CoRAL - Collaborative Review Analysis of Literature: visualization for views of collaborative relations of users in Fab4 browser
        \item[-] Movie Tracker: visualization of taste inference in Flixster.com, designed for data mining researchers
        \item[-] Replicated Distributed File System with Transitional Semantics: enabled user to read and modify files remotely, recovery mechanism is implemented
        \item[-] Span Space Render: implemented iso-surface extraction 
		\item[-] Ray Tracing Programming
		\item[-] Trust-Based Community Discovery for Hierarchical Collaborative Filtering: A new data mining algorithm for discovering the community in social network to provide better results for recommendation systesm
		\item[-] Agent of Backgammon Game: an AI agent to play backgammon
		\item[-] My Family Tree: an SNS family tree website on Django
	\end{itemize}

\section {Work Experience}
    {\sl Network System Support Intern} \hfill May 2011 - Dec 2011 \\
	Research Computing Group, IT Services, Simon Fraser University, Canada 
	\begin{itemize} \itemsep -2pt
	    \item[-] Developed group feature for Unified, Federated Development Service, implemented ten sets of script to better support the system, managed more than 1500 accounts
	\end{itemize}
	\vspace{-1.3em}
	{\sl Software Developer Intern} \hfill Jan 2012 - Apr 2012 (expected)\\
	Research In Motion (RIM), Canada
	\begin{itemize} \itemsep -2pt
        \item[-] BlackBerry Messager Software Developer
	\end{itemize}
	

\section {Scholarship and Awards}
    {\sl SFU Dean's Honour of Faculty of Applied Science} \hfill 2010 \\
	{\sl SFU Dr.Yu Award for Dual Degree Program Students} \hfill 2010 \\
	{\sl Entrance Award of SFU-ZU Dual Degree Program} \hfill 2009


\section {Extracurricular Activities}
    {\sl Volunteer} \hfill Sep 2010\\
	Grace Hopper Conference of Women in Computing, Atlanta, United States

	
\end{resume}
\end{document}


